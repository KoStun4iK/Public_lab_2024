\documentclass{beamer}
\usepackage{graphicx} % Required for inserting images
\usepackage[T2A]{fontenc}
\usepackage{amsmath}
%Hyphenation rules
%--------------------------------------
\usepackage{hyphenat}
\hyphenation{ма-те-ма-ти-ка вос-ста-нав-ли-вать}
%--------------------------------------
\usepackage[english, russian]{babel}
\usetheme{Madrid}
\title{История ОГУ им. И.С. Тургенева}
\author{К. А. Тарасов}
\institute{ОГУ им. И.С. Тургенева}
\date{29 апреля 2024}

\begin{document}

\frame{\titlepage}

\begin{frame}

\frametitle{Содержание}
\tableofcontents

\end{frame}
%--------------------------
\section{История университета}
%--------------------------
\begin{frame}\frametitle{История университета}
\begin{center}
\alert{Орло́вский госуда́рственный университе́т имени И. С. Тургенева (ОГУ)} – крупнейший вуз Орловской области и единственный в регионе многопрофильный центр, ведущий непрерывную подготовку специалистов всех уровней в области педагогического, инженерного, медицинского, естественнонаучного и гуманитарного образования.

В 2016 году ОГУ имени И. С. Тургенева стал одним из первых опорных университетов в России, в 2019 году отметил 100-летний юбилей, а в 2021 стал участником программы государственной поддержки вузов «Приоритет 2030».
Сегодня \alert{Орловский государственный университет им И.С. Тургенева} – это градообразующая организация региона, в которой учатся и работают более 18 тыс. человек. 
\end{center}
\begin{center}
\includegraphics{1024px-OGPI1966}
\end{center}
\end{frame}

\begin{frame}
\begin{center}
\frametitle{Ректоры}
\begin{tabular}{ |c|     c     | } 
 \hline
 Годы  & Ректор \\ 
 \hline
 1949—1954 & С. И. Ефремов \\
 \hline
 1978—1988 & Н. С. Антонов \\
 \hline
 1989—1992 & С. А. Пискунов \\
 \hline
 1992—2013 & Ф. С. Авдеев \\
 \hline
 2014—2015 & В. Ф. Ницевич \\
 \hline
 2015—2019 & О. В. Пилипенко \\
 \hline
 2019-н.в. & А. А. Федотов \\
 \hline
\end{tabular}
\end{center}
%--------------------------
\section{Структура университета}
%--------------------------

\end{frame}
\begin{frame}
\frametitle{Филиалы}
\begin{itemize}
  \item Ливенский филиал федерального государственного бюджетного образовательного учреждения высшего образования «Орловский государственный университет имени И.С. Тургенева»
  \centering
    \begin{minipage}{0.7\textwidth}
        \centering
  \includegraphics[width=\textwidth]{korpus_tn.jpg}
  \end{minipage}
\end{itemize}
\end{frame}

\begin{frame}
\frametitle{Филиалы}
\begin{itemize}
  \item Мценский филиал федерального государственного бюджетного
образовательного учреждения высшего образования «Орловский государственный университет имени И.С.Тургенева»
  \centering
    \begin{minipage}{0.7\textwidth}
        \centering
  \includegraphics[width=\textwidth]{корпус.jpg}
  \end{minipage}
\end{itemize}
\end{frame}

\begin{frame}
\frametitle{Филиалы}
\begin{itemize}
  \item Карачевский филиал федерального государственного бюджетного образовательного учреждения высшего образования
Орловский государственный университет
имени И.С. Тургенева
  \centering
    \begin{minipage}{0.7\textwidth}
        \centering
  \includegraphics[width=\textwidth]{karachev.jpg}
  \end{minipage}
\end{itemize}
\end{frame}

\begin{frame}{Продукты и услуги}

    \alert{Образование:}
    \begin{itemize}
    \item Подготовка к ОГЭ и ЕГЭ
    \item Юношеские специализированные научно-исследовательские школы
    \item Школы будущих профессий
    \item Получение среднего процессионального и высшего образования
    \item Дополнительное профессиональное образование
    \end{itemize}

    \alert{Наука и консалтинг:}
    \begin{itemize}
    \item Разработки университета. Приобретение и партнёрство
    \end{itemize}

    \alert{Отдых. Спорт. Здоровье:}
    \begin{itemize}
    \item База отдыха "Зелёный берег"
    \item Спорткомплекс "Спортивная арена"
    \item Студенческая поликилиника
    \item Крытый каток г. Мценск
    \end{itemize}
    
\end{frame}

\section{Физико-математический факультет}
\begin{frame}{О физико-математическом факультете, кафедра мат. анализа и диф. уравнений}
1 сентября 2022г. кафедра математического анализа и дифференциальных уравнений, созданная в 1977 году, была переименована, ее новое название "кафедра математического анализа и методики обучения математике". С 1988 г. заведующим кафедрой математического анализа и дифференциальных уравнений был доктор физико-математических наук, профессор, Заслуженный деятель науки РФ Александр Николаевич Зарубин. Основной задачей кафедры является обеспечение студентов достаточным уровнем знаний, умений и навыков в области математики с учетом направления подготовки, обучение применению полученных знаний в профессиональной деятельности; подготовка учителей по математике и физике (математике и информатике) . На новую кафедру переведены д.п.н., профессор Т.К.Авдеева, к.ф.-м.н, доцент И.С. Логунов, к.п.н, доцент Т.Л. Овсянникова
\end{frame}

\begin{frame}
\begin{center}
\frametitle{Преподаватели физико-математического факультета}
\begin{tabular}{ |c|c|c| } 
 \hline
 Учёная степень & Преподаватель & Контакты \\ 
 \hline
 доктор педагогических наук & Авдеева Т. К. & ivanavd@mail.ru \\
 \hline
 доктор педагогических наук & Тарасова Оксана Викторовна & 89103046617 \\
 \hline
 доктор педагогических наук & Аксёнов Андрей Александрович & +7 906 661 05 87 \\
 \hline
 кандидат физико-математических наук & Алексеева Елена Николаевна & +7(4862) 59-13-79 \\
 \hline
 кандидат физико-математических наук & Можарова Татьяна Николаевна &  8-920-287-85-91 \\
 \hline
 кандидат физико-математических наук & Соломатин Олег Дмитриевич & solomatinod@bk.ru \\
 \hline
 кандидат физико-математических наук & Чаплыгина Елена Викторовна & matdiff@yandex.ru \\
 \hline
 кандидат физико-математических наук & Панюшкин Сергей Владимирович & fauris@list.ru \\
 \hline
 кандидат педагогических наук & Овсянникова Татьяна Львовна & otl19@yandex.ru \\
 \hline
 кандидат физико-математических наук & Логунов Игорь Сергеевич & 8-4862-752425\\
 \hline
 кандидат физико-математических наук & Балахнев Максим Юрьевич & maxibal@yandex.ru \\
 \hline
 
\end{tabular}
\end{center}
\end{frame}
\end{document}



\begin{tabular}{ |c|c|c| } 
 \hline
 cell1 & cell2 & cell3 \\ 
 \hline
 cell4 & cell5 & cell6 \\ 
 \hline
 cell7 & cell8 & cell9 \\