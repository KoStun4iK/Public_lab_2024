\documentclass[a4paper,11pt]{article}
\usepackage[utf8]{inputenc}
\usepackage[russian]{babel}
\usepackage{geometry}
\usepackage{setspace}
\usepackage{graphicx}
\usepackage{caption}
\usepackage{amsmath}
\usepackage{array} % required for text wrapping in tables

\geometry{left=3cm, right=2cm, top=2cm, bottom=2cm}
\onehalfspacing

\begin{document}

\noindent
УДК 519.715

\begin{center}
\textbf{ПОСТРОЕНИЕ ГРАФА ПЕРЕХОДОВ \\ПОСЛЕДОВАТЕЛЬНОСТНОЙ СХЕМЫ \\ С ПРИМЕНЕНИЕМ SAT-РЕШАТЕЛЯ}

Иванов Иван Иванович\\
кандидат технических наук, доцент\\
доцент, Томский государственный университет\\
Россия, Томск\\
Петров Петр Петрович\\
студент, Томский государственный университет\\
Россия, Томск
\end{center}

\vspace{0.5cm}

\textbf{Аннотация.} Рассматривается подход к построению графа переходов последовательностной схемы. Исследуется метод, основанный на использовании SAT-решателя. Для построения графа переходов определяются возможные переходы между состояниями последовательной схемы и используются предварительные вычисления, основанные на троичном и двоичном моделированиях, значительно сокращающие объем вычислений. Также рассматривается построение на основе графа переходов последовательностной, обеспечивающий заданные переходы схемы. Компьютерные эксперименты показали эффективность предложенного метода построения графа переходов с применением SAT-решателя.

\textbf{Ключевые слова:} граф переходов, троичное моделирование, SAT-решатель, последовательностная схема, переходная последовательность. \\

В работе рассматривается подход к построению графа переходов синхронной последовательностной схемы. Построение графа, основанное на использовании SAT-решателя, исследуется подробно. Представлены результаты компьютерных экспериментов для предложенного метода построения графа переходов, применяемого SAT-решателя. Также кратко рассматривается решение задачи построения последовательности входных векторов, обеспечивающей перевод схемы в одно из состояний заданного множества, по графу переходов.

Рассмотрим синхронную последовательностную схему с {n} входами, m выходами, и r элементами памяти (триггерами).

\newpage
\noindent
\textit{X} = \{\textit{x}\textsubscript{1}, ..., \textit{x\textsubscript{n}}\} - входные переменные схемы, \textit{Y} = \{\textit{y}\textsubscript{1}, ..., \textit{y\textsubscript{m}}\} - ее выходные переменные, \textit{Z} = \{\textit{z}\textsubscript{1}, ..., \textit{z\textsubscript{p}}\} – внутренние переменные схемы.

Назовем \textit{графом переходов} \textit{последовательностной} \textit{схемы }ориентированный граф, у которого вершины сопоставлены состояниям схемы и есть дуга из вершины \textit{i} в вершину \textit{j} тогда и только тогда, когда в схеме существует одношаговый переход из состояния, соответствующего вершине \textit{i}, в состояние, соответствующее вершине \textit{j}, при каких-либо значениях входных переменных.

На рисунке 1 представлена комбинационная составляющая \textit{С} последовательностной схемы. При построении графа переходов рассматривается только та часть схемы, которая необходима для получения функций переходов, выходы схемы не рассматриваются. Поэтому структурное описание комбинационной составляющей, используемой для получения графа переходов, упрощается (рисунок 2). 

\begin{figure}[h]
    \centering
    \begin{minipage}{0.45\textwidth}
        \centering
        \includegraphics[width=\textwidth]{1.png}
        \caption*{Рисунок 1 --- Ирис setosa}
        \label{fig:1}
    \end{minipage}
    \hfill
    \begin{minipage}{0.45\textwidth}
        \centering
        \includegraphics[width=\textwidth]{2.png}
        \caption*{Рисунок 2 --- Ирис virginica}
        \label{fig:2}
    \end{minipage}
\end{figure}
В схеме с рисунка 2 можно исключить все элементы, не связанные с ее выходами, то есть с входами триггеров последовательностной схемы.

Система функций переходов последовательностной схемы имеет вид:
\begin{equation}
\begin{aligned}
z_j^t = \psi_j(x_1^t, \ldots, x_n^t, z_{1}^{t-1}, \ldots, z_{p}^{t-1}), \; j = \overline{1,p}
\end{aligned}
\tag{1}
\end{equation}

Будем представлять полное состояние схемы вектором (a,d), где $\alpha$ - вектор значений входных переменных \textit{X}, а $\delta$  - вектор значений внутренних переменных \textit{Z}.

\newpage

Двоичный вектор  $\tau^i = (\tau_{1}^i,...,\tau_{p}^i)$ значений переменных \textit{Z} будем называть \textit{кодом состояния} \textit{q\textsubscript{i}}. \textit{Q} = \{\textit{q}\textsubscript{1},...,\textit{q\textsubscript{t}}\}, где \textit{t }= 2\textit{\textsuperscript{p}}, – множество всех состояний схемы.

Рассмотрим общий подход к построению графа переходов последовательностной схемы предложенный в работах [1, 2] и других.

Представленные свойства определены в [3].

Рассмотрим подробнее следующее свойство, сформулированное в [3], используемое на 2-ом шаге сокращения вычислений.

\textit{Пусть выполнено точное троичное моделирование функций переходов системы }(1)\textit{ на векторе }($\alpha,\delta$)\textit{, представляющем полное состояние, и получен вектор значений внутренних переменных }$\delta^'$\textit{. }$\delta^'$\textit{ представляет минимальный покрывающий интервал множества булевых векторов значений переменных }\textit{Z}\textit{, а не точное множество этих векторов. Таким образом, множество состояний схемы достижимых за один шаг из множества }\textit{N}($\alpha,\delta$) \textit{может быть подмножеством множества }\textit{N}($\delta^'$)\textit{. }

\textbf{Результаты экспериментов}

Для проверки эффективности предложенного метода построения графа переходов последовательностной схемы, использующего SAT-решатель, были проведены эксперименты на бенчмарках. Эксперименты проводились на контрольных примерах (бенчмарках) ISCAS’89, представляющих последовательностные схемы. Для оценивания работы исследуемого метода при проведении экспериментов для различных бенчмарок измерялось время построения графа переходов, а также процент определяемых значений элементов матрицы \textit{M} на каждом шаге предварительных вычислений. Результаты компьютерных экспериментов представлены в таблице 1. 

\begin{center}
Таблица 1 – Результаты построения графов переходов.
\vspace{0.3cm}

\begin{tabular}{|>{\centering\arraybackslash}p{0.09\linewidth}|>{\centering\arraybackslash}p{0.09\linewidth}|>{\centering\arraybackslash}p{0.09\linewidth}|>{\centering\arraybackslash}p{0.09\linewidth}|>{\centering\arraybackslash}p{0.09\linewidth}|>{\centering\arraybackslash}p{0.09\linewidth}|>{\centering\arraybackslash}p{0.09\linewidth}|>{\centering\arraybackslash}p{0.09\linewidth}|>{\centering\arraybackslash}p{0.09\linewidth}|}
\hline
\textbf{Бенч-}

\textbf{марки} & \textbf{Входы}

\textbf{(}\textbf{\#}\textbf{)} & \textbf{Выходы}

\textbf{(}\textbf{\#}\textbf{)} & \textbf{Элемен-} \textbf{ты}

\textbf{памяти}

\textbf{(}\textbf{\#}\textbf{)} & \textbf{Элемен-} \textbf{ты}

\textbf{(}\textbf{\#}\textbf{)}

  & \textbf{Среднее время построения }

\textbf{графа}

\textbf{(сек.)}
& \textbf{Опреде-}\textbf{ленные}

\textbf{перехо-}\textbf{ды на шаге 1}

\textbf{(только 1)}

\textbf{(\%)} & \textbf{Опреде-}\textbf{ленные}

\textbf{перехо-}\textbf{ды на шаге 2}

\textbf{(только 0)}

\textbf{(\%)} & \textbf{Соотно-}\textbf{шение 0 и 1 в матрице }\textbf{\textit{M}}

\textbf{(\#0; \#1)} \\
\hline
S27 & 4 & 1 & 3 & 10 & 0,01 & 17,19 & 31,25 & 31; 33 \\
\hline
S386 & 7 & 7 & 6 & 159 & 1,01 & 2,15 & 92,77 & 3800; 296 \\
\hline
S832 & 18 & 19 & 5 & 287 & 7,60 & 11,13 & 61,23 & 710; 314 \\
\hline
S510 & 19 & 7 & 6 & 211 & 21,84 & 2,46 & 97,36 & 3995; 101 \\
\hline
S1488 & 8 & 19 & 6 & 653 & 4,82 & 3,13 & 82,25 & 3369; 727 \\
\hline

\end{tabular}
\end{center}

\vspace{0.3cm}

В таблице представлены следующие данные: имя бенчмарки, количество входов, выходов,
элементов памяти и элементов схемы;

\newpage

среднее время построения графа переходов схемы (по трем экспериментам); процент существующих одношаговых переходов в графе, определенных на шаге 1 предварительных вычислений, процент не существующих одношаговых переходов, определенных на шаге 2 предварительных вычислений, и соотношение 0 и 1 в полученной матрице \textit{M}. 

Шаги предварительных вычислений в экспериментах для рассмотренных бенчмарок определили от 48,44\% до 99,82\% одношаговых переходов (существующих и не существующих в графе). Шаг 2 предварительных вычислений, выполненный с помощью троичного моделирования, позволил определить большую часть одношаговых переходов, отсутствующих в графе. Значительная часть всех одношаговых переходов графа вычислена с помощью шагов 1 и 2 предварительных вычислений. Графы переходов для схем среднего размера построены за несколько секунд. 

 

\begin{center} 
\textbf{Список литературы}
\end{center}

 1. Иванов И.И. Построение графа переходов последовательностной схемы // Современные проблемы физико-математических наук: материалы IV Всероссийской науч.-практ. конф. с международным участием, (22 – 25 ноября 2018 г., г. Орёл). – Орел: ОГУ им. И.С. Тургенева, 2018. – Ч. 1. – С. 230 – 235. 

2. Ivanov I. Three-Value Simulation of Combinational and Sequential Circuits and its Applications // \href{https://ieeexplore.ieee.org/xpl/mostRecentIssue.jsp?punumber=8398195}{2020 IEEE Moscow Workshop on Electronic and Networking Technologies (MWENT)}. Moscow, Russia. 11−13 March 2020. – 7 pp.

3. Иванов И.И. Интервальные расширения булевых функций и троичное моделирование последовательностных схем // Таврический научный обозреватель. – 2017. – №5 (22). – С. 208-220.

4. Иванов И.И. Троичное моделирование комбинационных и синхронных последовательностных схем // Современные проблемы физико-математических наук / материалы V Всероссийской науч.-практ. конф. с международным участием, (26 – 29 сентября 2019 г., г. Орёл). – Орел: ОГУ им. И.С. Тургенева, 2019. – С. 274 – 281.


\end{document}
